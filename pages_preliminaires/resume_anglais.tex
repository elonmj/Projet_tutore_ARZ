\chapter*{Abstract}
\thispagestyle{empty}

This thesis presents an extension of the LWR (Lighthill-Whitham-Richards) macroscopic traffic model adapted to the specific characteristics of Benin's road network. The transportation system in Benin exhibits unique features: predominance of motorcycles (Zemidjans), heterogeneity of road infrastructure, and particular driving behaviors that are not properly represented by traditional models.

Our approach introduces a multi-class model explicitly integrating motorcycle behavior and their interaction with other vehicles. We define specific modulation functions $f_{M,i}(\rho_M)$ that quantify the impact of motorcycles on traffic flow, including gap-filling and interweaving phenomena. Additionally, we incorporate the variable influence of road surface quality on the speed of different vehicle classes through a slowing coefficient $\lambda_{\text{mat},i}$.

The model is calibrated and validated using real data collected from Benin's road network. Results demonstrate its ability to faithfully represent local dynamics, particularly at intersections and in congested areas. A sensitivity analysis identifies the most influential parameters and quantifies the associated uncertainty.

This extension of the LWR model provides a valuable tool for traffic planning and transportation policy evaluation in Benin, accounting for the complex reality of local traffic dominated by two-wheeled motorized vehicles.

\vspace{1cm}

\noindent \textbf{Keywords:} traffic modeling, macroscopic approach, LWR model, multi-class traffic, motorcycles, Benin, gap-filling, interweaving, slowing coefficient, calibration, validation.

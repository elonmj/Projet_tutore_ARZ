\chapter*{Résumé}
\thispagestyle{empty}

Ce mémoire propose une extension du modèle macroscopique de trafic LWR (Lighthill-Whitham-Richards) adaptée aux spécificités du réseau routier béninois. Le système de transport au Bénin présente des caractéristiques uniques : prédominance des motos (Zémidjans), hétérogénéité des infrastructures routières, et comportements de conduite particuliers qui ne sont pas correctement représentés par les modèles traditionnels.

Notre approche introduit un modèle multiclasses intégrant explicitement le comportement des motos et leur interaction avec les autres véhicules. Nous définissons des fonctions de modulation spécifiques $f_{M,i}(\rho_M)$ qui quantifient l'impact des motos sur l'écoulement du trafic, notamment les phénomènes de gap-filling et d'interweaving. De plus, nous intégrons l'influence variable du revêtement routier sur la vitesse des différentes classes de véhicules à travers un coefficient de ralentissement $\lambda_{\text{mat},i}$.

Le modèle est calibré et validé sur des données réelles collectées sur le réseau routier béninois. Les résultats démontrent sa capacité à représenter fidèlement les dynamiques locales, notamment aux intersections et dans les zones de congestion. Une analyse de sensibilité identifie les paramètres les plus influents et quantifie l'incertitude associée.

Cette extension du modèle LWR constitue un outil précieux pour la planification du trafic et l'évaluation des politiques de transport au Bénin, prenant en compte la réalité complexe de la circulation locale dominée par les deux-roues motorisés.

\vspace{1cm}

\noindent \textbf{Mots-clés :} modélisation du trafic, approche macroscopique, modèle LWR, trafic multiclasses, motos, Bénin, gap-filling, interweaving, coefficient de ralentissement, calibration, validation.

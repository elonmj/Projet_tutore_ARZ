% --- Core Packages ---
% \usepackage[utf8]{inputenc}         % Not needed with Xe/LuaLaTeX + fontspec
% \usepackage[T1]{fontenc}            % fontspec handles this
% \usepackage{lmodern}              % Replaced by fontspec definitions
\usepackage{fontspec}             % Use if compiling with XeLaTeX or LuaLaTeX for wider font choices

% --- Language and Babel ---
\usepackage[english,french,shorthands=off]{babel} % Support for French and English, disable French shorthands to avoid conflicts
\usepackage{csquotes}               % Context-sensitive quotation marks (recommended with biblatex)

% --- Mathematics ---
\usepackage{amsmath}                % AMS math enhancements
\usepackage{amssymb}                % AMS symbols
% \usepackage{amsthm}                 % Theorem environments (Replaced by ntheorem)
\usepackage{mathtools}              % Further math enhancements (e.g., \coloneqq)
\usepackage{bm}                     % Bold math symbols (\bm{})

% --- Layout and Appearance ---
\usepackage{geometry}               % Page layout (configured in layout.tex)
\usepackage{graphicx}               % Include graphics (\includegraphics)
\usepackage{xcolor}                 % Define and use colors (configured in colors.tex)
\usepackage{setspace}               % Control line spacing (e.g., \onehalfspacing)
\usepackage{titlesec}               % Customize chapter/section titles (configured in typography.tex)
\usepackage{microtype}              % Improves typography (subtle adjustments to spacing, protrusion) - Highly recommended
\usepackage{booktabs}               % Nicer tables (\toprule, \midrule, \bottomrule)
\usepackage{caption}                % Customize captions for figures and tables
\usepackage{subcaption}             % Support for subfigures and subtables
\usepackage{fancyhdr}               % Custom headers and footers (configured in layout.tex)
\usepackage{enumitem}               % Customize lists (configured in typography.tex)
\usepackage[framemethod=TikZ]{mdframed} % Framed boxes, needed for ntheorem style
\usepackage[amsmath,thmmarks,hyperref]{ntheorem} % Theorem environments (configured in theorems.tex) - amsmath needed for compatibility
\usepackage{etoolbox}               % Programming tools (used for spacing around environments)

% --- References and Links ---
\usepackage[
    backend=biber,              % Use biber backend (more powerful than bibtex)
    style=alphabetic,              % Citation style (e.g., numeric, authoryear, alphabetic) - Choose based on requirements
    sorting=ynt,                % Sorting order (year,name, title)
    citestyle=alphabetic,     % Compact numeric citations (e.g., [1-3])
    maxbibnames=99,             % Show all authors in bibliography
    maxcitenames=2,             % Show max 2 authors in citation before using 'et al.'
    giveninits=true,            % Use initials for given names
    uniquename=init            % Disambiguate authors with same last name using initials
]{biblatex}                 % Modern bibliography management
\usepackage[colorlinks=true, linkcolor=blue, citecolor=green, urlcolor=magenta]{hyperref} % Clickable links and references
\usepackage[capitalise,noabbrev]{cleveref} % Smarter cross-referencing (\cref{}, \Cref{}) - Must be loaded AFTER hyperref and ntheorem

% --- Plotting ---
\usepackage{pgfplots}              % Plotting package
\pgfplotsset{compat=newest}       % Use newest features
\pgfplotsset{ % Apply style from benintraffic.sty
    compat=newest,
    every axis/.append style={
        line width=0.5pt,
        tick style={color=midgray, line width=0.5pt} % midgray needs to be defined in colors.tex
    }
}
\usetikzlibrary{arrows.meta}      % For newer arrow tip styles if needed

% --- Utilities ---
\usepackage{lipsum}                 % Generate dummy text (for testing layout)
\usepackage{todonotes}              % Add TODO notes in the margin
\usepackage{import}                 % Allows importing files from other directories (\subimport, \import)

% --- Glossary ---

\usepackage[acronym, nonumberlist, toc, section=chapter]{glossaries-extra} % For acronyms and glossary
% --- Graphics Path ---
\makenoidxglossaries % Initialize glossaries
% Define where LaTeX should look for graphics files
\graphicspath{{assets/}} % Relative path from main.tex to assets folder
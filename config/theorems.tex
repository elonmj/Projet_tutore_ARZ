% Define theorem-like environments using ntheorem and mdframed
% Requires ntheorem, mdframed, xcolor packages (loaded in packages.tex)
% Requires color definitions (darkblue, lightblue, theorem*) from colors.tex

% % --- Basic Theorem Styles ---
% % Option 1: Standard amsthm styles
% \theoremstyle{plain} % Style for theorems, lemmas, propositions, corollaries (italic text)
% \newtheorem{theorem}{Theorem}[section] % Numbered within section
% \newtheorem{lemma}[theorem]{Lemma}     % Use same counter as theorem
% \newtheorem{proposition}[theorem]{Proposition}
% \newtheorem{corollary}[theorem]{Corollary}

% \theoremstyle{definition} % Style for definitions, conditions, problems, examples (upright text)
% \newtheorem{definition}[theorem]{Definition}
% \newtheorem{condition}[theorem]{Condition}
% \newtheorem{example}[theorem]{Example}
% \newtheorem{problem}[theorem]{Problem}
% \newtheorem{assumption}[theorem]{Assumption} % Added Assumption environment

% \theoremstyle{remark} % Style for remarks, notes, claims, summaries, cases (upright text, less prominent heading)
% \newtheorem{remark}[theorem]{Remark}
% \newtheorem{note}[theorem]{Note}
% \newtheorem{claim}[theorem]{Claim}
% \newtheorem{summary}[theorem]{Summary}
% \newtheorem{case}[theorem]{Case}

% % --- Customization Example (Optional - using mdframed or tcolorbox) ---
% % For more visual customization (e.g., colored boxes), you might use packages
% % like mdframed or tcolorbox. This requires adding the chosen package to packages.tex.
% % Below is a *conceptual example* using pseudo-commands. You'd need to implement
% % this properly with the actual package syntax if desired.

% % \usepackage{mdframed} % Add this to packages.tex if using mdframed

% % Example of a styled theorem box (requires package and proper setup)
% % \newmdtheoremenv[
% %   linecolor=darkblue,
% %   linewidth=1pt,
% %   topline=true, bottomline=true, leftline=true, rightline=true,
% %   backgroundcolor=theorembodyshade,
% %   frametitlebackgroundcolor=theoremtitleshade,
% %   frametitlerule=true,
% %   frametitlefont=\bfseries,
% %   innertopmargin=\topskip,
% % ]{stheorem}{Theorem}[section]

% % \newmdtheoremenv[
% %   style=stheorem, % Inherit style from stheorem
% % ]{slemma}[stheorem]{Lemma} % Use same counter

% --- Theorem Style Definition (using mdframed via ntheorem) ---
\theoremstyle{nonumberplain} % Base style for frame setup
\theoremheaderfont{\normalfont\bfseries\color{theoremheadfontcolor}}
\theorembodyfont{\normalfont\itshape} % Italic body for theorems
\theoremseparator{:}
\theorempreskip{12pt} % Space above
\theorempostskip{12pt} % Space below

\newmdenv[
    linewidth=1pt,
    linecolor=theoremlincolor,
    backgroundcolor=theorembackground,
    roundcorner=5pt,
    skipabove=\theorempreskip, % Use ntheorem spacing variables
    skipbelow=\theorempostskip
]{theorembox}

\def\theoremframecommand{\theorembox} % Apply this frame to environments using theoremstyle plain
\def\lemmaframecommand{\theorembox}
\def\propositionframecommand{\theorembox}
\def\corollaryframecommand{\theorembox}

\theoremstyle{plain} % Apply the frame settings defined above
\theoremsymbol{} % No symbol by default
\theoremheaderfont{\normalfont\bfseries\color{theoremheadfontcolor}}
\theorembodyfont{\normalfont\itshape}
\theoremseparator{:}
\theorempreskip{12pt}
\theorempostskip{12pt}

\newtheorem{theorem}{Théorème}[chapter] % Numbered within chapter for book class
\newtheorem{proposition}[theorem]{Proposition}
\newtheorem{lemma}[theorem]{Lemme}
\newtheorem{corollary}[theorem]{Corollaire}

% --- Simpler Definition Style ---
\theoremstyle{plain} % Still use plain, but customize fonts/spacing
\theoremsymbol{}
\theoremheaderfont{\normalfont\bfseries} % Simpler header font
\theorembodyfont{\normalfont} % Upright body font
\theoremseparator{:}
\theorempreskip{6pt} % Less space above/below
\theorempostskip{6pt}
\def\definitionframecommand{} % No frame for definitions
\def\remarkframecommand{}
\def\exampleframecommand{}

\newtheorem{definition}[theorem]{Définition} % Use same counter as theorem
\newtheorem{remark}[theorem]{Remarque}
\newtheorem{example}[theorem]{Exemple}
\newtheorem{assumption}[theorem]{Hypothèse} % Added Assumption

% --- Cleveref Configuration ---
% Define how \cref should refer to these environments
\crefname{theorem}{Theorem}{Theorems}
\Crefname{theorem}{Theorem}{Theorems}
\crefname{lemma}{Lemma}{Lemmas}
\Crefname{lemma}{Lemma}{Lemmas}
\crefname{proposition}{Proposition}{Propositions}
\Crefname{proposition}{Proposition}{Propositions}
\crefname{corollary}{Corollary}{Corollaries}
\Crefname{corollary}{Corollary}{Corollaries}
\crefname{definition}{Definition}{Definitions}
\Crefname{definition}{Definition}{Definitions}
\crefname{example}{Example}{Examples}
\Crefname{example}{Example}{Examples}
\crefname{remark}{Remark}{Remarks}
\Crefname{remark}{Remark}{Remarks}
\crefname{assumption}{Assumption}{Assumptions}
\Crefname{assumption}{Assumption}{Assumptions}
\crefname{section}{Section}{Sections} % Also configure for sections, figures, tables etc.
\Crefname{section}{Section}{Sections}

% Define how \cref should refer to these environments (French names)
\crefname{theorem}{Théorème}{Théorèmes}
\Crefname{theorem}{Théorème}{Théorèmes}
\crefname{lemma}{Lemme}{Lemmes}
\Crefname{lemma}{Lemme}{Lemmes}
\crefname{proposition}{Proposition}{Propositions}
\Crefname{proposition}{Proposition}{Propositions}
\crefname{corollary}{Corollaire}{Corollaires}
\Crefname{corollary}{Corollaire}{Corollaires}
\crefname{definition}{Définition}{Définitions}
\Crefname{definition}{Définition}{Définitions}
\crefname{example}{Exemple}{Exemples}
\Crefname{example}{Exemple}{Exemples}
\crefname{remark}{Remarque}{Remarques}
\Crefname{remark}{Remarque}{Remarques}
\crefname{assumption}{Hypothèse}{Hypothèses}
\Crefname{assumption}{Hypothèse}{Hypothèses}

% Names for standard elements (ensure consistency with babel french)
\crefname{chapter}{Chapitre}{Chapitres}
\Crefname{chapter}{Chapitre}{Chapitres}
\crefname{section}{Section}{Sections}
\Crefname{section}{Section}{Sections}
\crefname{figure}{Figure}{Figures}
\Crefname{figure}{Figure}{Figures}
\crefname{table}{Table}{Tables}
\Crefname{table}{Table}{Tables}
\crefname{equation}{Equation}{Equations} % Default is usually 'eq.'/'eqs.'
\Crefname{equation}{Equation}{Equations} % Default is usually 'Eq.'/'Eqs.'